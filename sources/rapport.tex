\documentclass[a4paper,11pt,final,oneside]{article}
% Pour une impression recto verso, utilisez plutôt ce documentclass :
%\documentclass[a4paper,11pt,twoside,final]{article}

\usepackage[english,francais]{babel}
\usepackage[utf8]{inputenc}
\usepackage[T1]{fontenc}
\usepackage[pdftex]{graphicx}
\usepackage{setspace}
\usepackage{hyperref}
\usepackage[french]{varioref}
\usepackage{geometry}
\usepackage{fancyhdr}
\usepackage{amsthm}
\usepackage[autostyle]{csquotes}
\usepackage{color}
\usepackage{changepage}
\usepackage{listings,xcolor}
\usepackage{inconsolata}
\usepackage{float}

\definecolor{dkgreen}{rgb}{0,.6,0}
\definecolor{dkblue}{rgb}{0,0,.6}
\definecolor{dkyellow}{cmyk}{0,0,.8,.3}

\lstset{
  language        = php,
  basicstyle      = \small\ttfamily,
  keywordstyle    = \color{dkblue},
  stringstyle     = \color{red},
  identifierstyle = \color{dkgreen},
  commentstyle    = \color{gray},
  emph            =[1]{php},
  emphstyle       =[1]\color{black},
  emph            =[2]{if,and,or,else},
  emphstyle       =[2]\color{dkyellow},
  literate=%
    {á}{{\'a}}1
    {à}{{\`a}}1
    {é}{{\'e}}1
    {è}{{\`e}}1
    {í}{{\'i}}1
    {ó}{{\'o}}1
    {ú}{{\'u}}1}

\pagestyle{fancy}
\lhead{Thibaud COURTOISON}
\rhead{Rapport de stage de fin de DUT}

\geometry{hmargin=2.8cm,vmargin=2.8cm}

\newcommand{\reporttitle}{Rapport de Stage de fin de DUT}     % Titre
\newcommand{\reportauthor}{Thibaud \textsc{Courtoison}} % Auteur
\newcommand{\reportsubject}{Stage de fin d'étude} % Sujet
\newcommand{\HRule}{\rule{\linewidth}{0.5mm}}
\setlength{\parskip}{1ex} % Espace entre les paragraphes

%for quoting
\newenvironment{aquote}[1]{
  \pushQED{#1}
  \begin{quotation}
  \it
  }{
  \par\nointerlineskip\noindent\hfill(\popQED)
  \end{quotation}
}

%Nota Bene for me
\newcommand{\nb}[1]{\textcolor{red}{#1}}

%Better cite command
\newcommand{\rf}[1]{\up{\textcolor{blue}{\cite{#1}}}}

%For corrections and notes by ronan
\newcommand{\ronan}[1]{\textcolor{cyan}{#1 - Ronan}}

%For blue URL
\newcommand{\blurl}[2]{\textcolor{blue}{\href{#1}{#2}}}

\hypersetup{
    pdftitle={\reporttitle},%
    pdfauthor={\reportauthor},%
    pdfsubject={\reportsubject},%
    pdfkeywords={rapport} {vos} {mots} {clés}
}

\graphicspath{{../img/}}

\begin{document}
  % Inspiré de http://en.wikibooks.org/wiki/LaTeX/Title_Creation

\begin{titlepage}

\begin{center}

% Images entreprise & association
\begin{minipage}[t]{0.48\textwidth}
  \begin{flushleft}
    \includegraphics [height=30mm]{logo-libertic.png} \\[0.2cm]
    \begin{spacing}{1.5}
      \textsc{\LARGE Association\\LiberTIC}
    \end{spacing}
  \end{flushleft}
\end{minipage}
\begin{minipage}[t]{0.48\textwidth}
  \begin{flushright}
    \includegraphics [height=30mm]{logo-polypodes.png} \\[0.5cm]
    \textsc{\LARGE Entreprise\\Les Polypodes}
  \end{flushright}
\end{minipage} %\\[0.8cm]

\vfill

% Titre Rapport
\textsc{\Large \reportsubject}\\[0.5cm]
\HRule \\[0.6cm]
{\huge \bfseries \reporttitle}\\[0.4cm]
\HRule \\[1.0cm]

% Auteur et encadrant
\begin{minipage}[t]{0.3\textwidth}
  \begin{flushleft} \large
    \emph{Auteur :}\\
    \reportauthor
  \end{flushleft}
\end{minipage}
\begin{minipage}[t]{0.6\textwidth}
  \begin{flushright} \large
    \emph{Responsables :} \\
    M.~Ronan \textsc{Guilloux} \\
    M.~Nicolas \textsc{Hernandez}
  \end{flushright}
\end{minipage}


\vfill

\begin{minipage}[t]{0.48\textwidth}

  \begin{flushleft}
    \includegraphics [height=30mm]{logo-univ.jpg} \\[0.2cm]
    \begin{spacing}{1.5}
      \textsc{\LARGE Université de Nantes}
    \end{spacing}
  \end{flushleft}

\end{minipage}
\begin{minipage}[t]{0.48\textwidth}

  \begin{flushright}
    \includegraphics [height=30mm]{logo-iut.jpg} \\[0.2cm]
    \begin{spacing}{1.5}
      \textsc{\LARGE IUT de Nantes}
    \end{spacing}
  \end{flushright}

\end{minipage}
%\vfill

{\large Du 13 avril 2015 au 19 juin 2015}

\end{center}

\end{titlepage}

  \cleardoublepage % Dans le cas du recto verso, ajoute une page blanche si besoin
  \sloppy          % Justification moins stricte : des mots ne dépasseront pas des paragraphes

  \part{Introduction}
  
  \section*{Remerciements}
\addcontentsline{toc}{section}{Remerciements}

Je tiens tout d'abord à remercier les personnes qui m'ont accompagnés durant ces deux années de DUT.\\

\textit{M. Ronan Guilloux} et toute l'équipe des Polypodes, pour m'avoir accordé leurs confiance en m'acceptant dans leur équipe et pour m'avoir accompagné durant ces deux mois et demi. Leur patience et leur aide ont fait de ce stage une expérience particulièrement constructive.\\

Et bien sûr, toute l'équipe pédagogique du département informatique de l'IUT de Nantes, qui m'ont fait découvrir les mondes passionnants de l'informatique, de la gestion, du droit et de la communication. Si ces deux années peuvent s'assimiler à une réussite, c'est en très grande partie grâce à eux.\\

\nb{Refact this}

  \cleardoublepage
  \part*{Resumé du rapport}
\addcontentsline{toc}{section}{Resumé}

\section*{Résumé français}

Ce document est le rapport de mon expérience de deux mois et demi dans l'élaboration du projet Open Data Event en collaboration avec LiberTIC, une association nantaise militant dans l'Open Data, et Les Polypodes, une agence web promouvant le Logiciel Libre, situé sur l'Île de Nantes. Vous trouverez dans un premier temps la présentation du stage et du projet. La suite vous présentera la phase d'Analyse et de Conception que j'ai effectué. Puis, je décrirais la phase de Développement avec les différentes technologies que j'ai pu aborder. Enfin, une partie présentera les possibilités de suite du projet. Et dernièrement, je ferai un bilan du projet, ainsi qu'un bilan personnel.

\section*{English abstract}

This document is a repport of my two month and a half experience in the making of the Open Data Event project in collaboration with LiberTIC, a Nantes-based organization activist in the Open Data field, and Les Polypodes, a web agency promoting Free Software, located at L'Île de Nantes. First, you will find a presentation of the internship and the projet. The following will present the analysis and design phases I made. Then, I will describe the development stage with different technologies I had the possibility to tackle. Finally, a section will present the probable future of the project. And lately, I will do a review of the project and a personnal assessment.
  \cleardoublepage
  \addcontentsline{toc}{section}{Sommaire}

\tableofcontents 
  \cleardoublepage
  \section{Le projet}

Une des particularités de mon stage est le fait que je travaillais en collaboration direct non seulement avec l'entreprise accompagnait mon stage, mais aussi avec des réprésentants de l'association LiberTIC, clients du projet. Cette section présentera donc Les Polypodes et LiberTIC, puis, je parlerais du projet, de ses premières versions et enfin de ma place sur ce projet.

\subsection{Données Ouvertes et LiberTIC}

\begin{aquote}{Wikipédia}
``L'ouverture des données (en anglais open data) représente à la fois un mouvement, une philosophie d'accès à l'information et une pratique de publication de données librement accessibles et exploitable.''
\end{aquote}

\subsection{Logiciel Libre et Les Polypodes}

\begin{aquote}{Wikipédia}
``Un logiciel libre est un logiciel dont l'utilisation, l'étude, la modification et la duplication en vue de sa diffusion sont permises, techniquement et légalement.''
\end{aquote}

\subsection{L'aggrégateur d'événement (ODE)}

\subsection{Les premières versions}

\subsection{Ma place sur le projet}
  \cleardoublepage

  \addtocontents{toc}{\protect\vspace{40pt}}

  \part{Développement}

  \section{Commencement: Analyse et Conception}

\subsection{Mise en place de mon environnement de travail}

\subsection{Mise à niveau préliminaire}

\subsection{Analyse des besoins}

\subsection{Conception et première réunion}
  \cleardoublepage
  \section{Développement et Tests}

\subsection{Symfony 2}

\subsection{Intégration de SabreDAV}

\subsection{Implémentation ElasticSearch}

\subsection{Implémentation de PostgreSQL}

\subsection{Fondation d'une API REST}

\nb{Ne pas oublier de citer: restsymfony}

\subsection{Tests unitaires}

\subsection{Tests de comportement}
  \cleardoublepage

  \addtocontents{toc}{\protect\vspace{40pt}}

  \part{Conclusion}

  \section{En marche vers ODE v2.1}

Même si j'ai rempli un partie importante du cahier des charges et j'ai effectué toutes les tâches qui m'ont été confiées au début de mon stage, \textbf{Open Data Event} doit encore évoluer pour atteindre un niveau de maturité et de fonctionnalité suffisant pour être proposé au grand public ou pour être intégrer à d'autres projets.

\subsection{Finir l'API}

Même si toutes les fonctionnalités demandées de l'API ont été développées, je n'ai pas eu le temps d'implémenter la gestion de l'authentification de l'API. Il n'est pas possible, en l'état, de proposer le service au grand public sans authentification.

\subsection{Import / Export}

Pour permettre à des organismes de faire une migration vers ODE, il faut pouvoir proposer un import simple des données, sous plusieurs formats (.ics,.xml,.csv). De plus, il est possible d'imaginer un import automatique des données depuis les sites des fournisseurs d'événements.

\subsection{Formulaires}

ODE v2.1 devrait proposer une customisation simplifié du formulaire de création d'événement. En effet, chaque instance d'ODE aura des besoins spécifiques quant à sa description d'un événement.
Une des possibilités serait de mettre à disposition un Schéma sémantique qui serait analysé pour créer un formulaire adapté.

De plus, il serait interessant de pouvoir intégrer un service de géolocalisation permettant à chaque événement d'être situé sur une carte. Cependant, il serait important de laisser la possibilité le service de reverse-geocoding de son choix, pour promouvoir les services libres tels que Open Street Map et ne pas obliger l'utilisation de Google Map.

Enfin, utiliser une auto-complétion sur les champs de Categorie et de Tags ne serait pas de trop pour ne pas multiplier les données inutilement.

\subsection{Brouillons}

Actuellement, dès la création d'un événement, il se retrouve en accès publique. Il serait intéressant de pouvoir créer des `` brouillons '' d'événements pour laisser le temps de remplir tous les champs de l'événement.

\subsection{Autres fonctionnalités}

Enfin, il y a de nombreuses autres fonctionnalités intéressantes à intégrer au projet, que cela soit l'authentification à OAuth2, l'utilisation de ReCaptcha et d'autres.
  \cleardoublepage
  \section*{Conclusion}
\addcontentsline{toc}{section}{Conclusion}

\subsection*{Bilan du projet}
\addcontentsline{toc}{subsection}{Bilan du projet}

\subsection*{Bilan personnel}
\addcontentsline{toc}{subsection}{Bilan personnel}
  \cleardoublepage
  \phantomsection\addcontentsline{toc}{section}{Références}
\begin{thebibliography}{ABC}
    \bibitem{cartopendata} LiberTIC. \emph{OpenData Map}. URL: \url{http://www.opendata-map.org/}. Consulté le 1 juin 2015.
    \bibitem{gitimmersion} Gitimmersion. \emph{Best GIT tuto ever -50 steps}. URL: \url{http://gitimmersion.com/}. Consulté le 13 avril 2015.
    \bibitem{successfulgit} Vincent Driessen. \emph{A successful Git branching model}. Janvier 2010. URL: \url{http://nvie.com/posts/a-successful-git-branching-model/}. Consulté le 13 avril 2015.
    \bibitem{bestpracticephp} Brian Fenton. \emph{Best Practices for Modern PHP Development}. URL: \url{https://www.airpair.com/php/posts/best-practices-for-modern-php-development}. Consulté le 14 avril 2015.
    \bibitem{stupidvssolid} Hugo Hamon. \emph{Votre code est STUPID ? Rendez le SOLID}. Décembre 2013. URL: \url{http://afsy.fr/avent/2013/02-principes-stupid-solid-poo}. Consulté le 14 avril 2015.
    \bibitem{symfonybook} Sensio Labs. \emph{The Symfony Book}. URL: \url{http://symfony.com/doc/current/book/index.html}. Consulté plusieurs fois du 15 avril 2015 au 19 juin 2015.
    \bibitem{restsymfony} William Durand. \emph{REST APIs with Symfony2: The Right Way}. Août 2012. URL: \url{http://williamdurand.fr/2012/08/02/rest-apis-with-symfony2-the-right-way/}. Consulté le 22 mai 2015.
\end{thebibliography}

  \cleardoublepage
  \section*{Annexes}
\phantomsection\addcontentsline{toc}{section}{Annexes}

Blablabla
\end{document}