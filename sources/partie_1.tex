\section{Le projet}

Une des particularités de mon stage est le fait que je travaillais en collaboration direct non seulement avec l'entreprise qui accompagnait mon stage, mais aussi avec des réprésentants de l'association LiberTIC, clients du projet. Cette section présentera donc Les Polypodes et LiberTIC, puis, je parlerais du projet, de ses premières versions et enfin de ma place sur ce projet.

\subsection{Données Ouvertes et LiberTIC}

\begin{aquote}{Wikipédia}
``L'ouverture des données (en anglais open data) représente à la fois un mouvement, une philosophie d'accès à l'information et une pratique de publication de données librement accessibles et exploitable.''
\end{aquote}

Il s'agit là d'une définition plutôt abstraite de l'ouverture des données. Mais pour résumé, on pourrait regrouper la définition en trois points:

\begin{itemize}
    \item[$\bullet$] \textbf{Disponibilité et Accès}: Les données doivent être disponible dans leurs ensembles et gratuitement. De plus, les données doivent être disponible dans un format libre et modifiable.
    \item[$\bullet$] \textbf{Réutilisation et Redistribution}: Les données doivent être disponibles sous une licence autorisant la réutilisation et la redistribution des données
    \item[$\bullet$] \textbf{Participation universelle}: Tout le monde doit être en mesure d'utiliser, de réutiliser et de redistribuer sans discrimination contre des personnes ou des groupes. (Exemple: restriction non-commerciale ne fait pas partie de l'open data)
\end{itemize}

On pourrait aussi définir l'Open Data comme la tendance à considérer l'information publique comme un bien commun dont la diffusion est d'intérêt public et général.

Grâce à des mouvements citoyens comme \textbf{Wikimédia France}, \textbf{Open Street Map France} ou \textbf{LiberTIC}, l'ouverture des données s'est popularisé en France. Ainsi, depuis quelques années, nous pouvons voir l'apparition de plusieurs portails de données de villes, de département ou de régions (tels que \textbf{Paris Data}, \textbf{Data Pays de la Loire}, etc...)

\nb{Parler un peu plus de ma relation avec LiberTIC ?}

\textbf{LiberTIC} est une association nantaise, fondé en 2009, de promotion de la Culture Libre et des Données Ouvertes dans l'espace francophone. Elle est particulièrement identifiée comme l'une des principales associations françaises œuvrant pour la promotion de l'Open Data.

LiberTIC publie notamment une carte de France de l'Open Data\rf{cartopendata} identifiant les projets de collectivités publiques, en cours ou déjà réalisés, de mise à disposition de données publiques transversales.

Durant mon stage, j'ai pu rencontrer plusieurs personnes de l'association, tels que:

\nb{Parler de Loïc et Aleth}

\subsection{Logiciel Libre et Les Polypodes}

\begin{aquote}{Wikipédia}
``Un logiciel libre est un logiciel dont l'utilisation, l'étude, la modification et la duplication en vue de sa diffusion sont permises, techniquement et légalement.''
\end{aquote}

D'une manière un peu simplifié, on peux dire qu'un logiciel libre est un logiciel qui peut être \textbf{utilisé}, \textbf{modifié} et \textbf{redistribué} sans restriction par la personne à qui il a été distribué. Un tel logiciel est ainsi susceptible d'être soumis à l'étude, critique et correction. C'est pourquoi les logiciels libres sont d'une certaines fiabilité et réactivité.

Des exemples de logiciels libres célèbres sont \textbf{Mozilla Firefox}, \textbf{Mozilla Thunderbird}, \textbf{OpenOffice} et \textbf{VLC}.

La première question que l'on se pose quand on entend parler de logiciel libre est la suivante:

\textit{``Mais si c'est gratuit, comment ils font pour gagner de l'argent??''}

Il s'agit la plupart du temps d'une confusion entre le développement d'un logiciel libre par une entreprise et du service proposé sur ce service par cette entreprise. Un des exemple intéressant est celui de \textbf{Canonical Ltd.}, le créateur d'Ubuntu. En effet, avec un système d'exploitation libre, mis à jour régulièrement, Canonical propose ses services aux entreprises pour l'installation, la customisation et l'utilisation de leur système d'exploitation. Ainsi, Ubuntu en tant que logiciel libre leur donne une visibilité sur le plan international et une crédibilité dans le milieu des systèmes d'exploitation.

C'est pourquoi de nombreuses entreprises se lance dans le logiciel libre, que cela soit par la diffusion de leurs propres logiciels sous licences libres ou par le maintien de d'autres logiciels libres.

\begin{aquote}{Wikipédia}
``Le Polypode commun (Polypodium vulgare L.) est une fougère de la famille des Polypodiaceae. Il est parfois appelé réglisse des bois ou réglisse sauvage. En effet, son rhizome a été utilisé à des fins médicinales, mais aussi gastronomiques.''
\end{aquote}

\textbf{Les Polypodes} est une agence web créé en 2005 par \textit{M. Antonio-Manuel FIDALGO} et qui emploie actuellement 8 salariés. Les bureaux de l'entreprise sont situés au 7\up{ème} étage de l'immeuble \textit{Sigma 2000}.

\nb{I need more informations}

\nb{Parler de l'implication des polypodes dans le logiciel libre}

\nb{Parler des profils avec qui j'ai pu travailler}

\subsection{L'aggrégateur d'événement (ODE)}

L'aggrégateur d'événement, appellé \textbf{Open Data Event} (ODE), est une plateforme permettant de relier des fournisseurs de d'événements  (les organisateurs de festivals, concerts, conférence, etc...) avec les réutilisateurs d’événements (les utilisateurs de l’API, les journaux, etc...). Le but principal de cette association est d'améliorer la diffusion des données.

\nb{J'ai besoin d'une meilleure définition...}

\subsection{Historique des versions}

ODE est un projet mené par LiberTIC commencé au début de 2014. Il est passé par plusieurs versions, traversant plusieurs phases d'analyses, de conceptions et de développements.

\subsubsection*{La première réunion}

L'élément déclencheur de ce projet est la réunion du 11 juin 2014 à Stéréolux avec l'association LiberTIC et quelques acteurs de l'événementiel à Nantes. Il est apparu qu'il y avait une nécessité d'améliorer la diffusion des données.

C'est ainsi qu'est né le projet ODE (Open Data Event). Il s'agit d'un aggrégateur d'événement permettant de relier les fournisseurs d'événements (les organisateurs de festivals, concerts, conférence, etc...) avec les réutilisateurs d'événements (les utilisateurs de l'API, les journaux, etc...).

Cette réunion a permis de produire un cahier des charges pour le projet (disponible à cette addresse: \url{https://github.com/LiberTIC/ODEV2/blob/master/doc/Documents/120622_ODE_cahierDesCharges_MakinaCorpus.pdf} )

\subsubsection*{Version 1 - Makina Corpus}

Makina Corpus ( \url{http://makina-corpus.com/} ), entreprise de développement de logiciels libres, a répondu au cahier des charges en proposant de réaliser le projet. Plusieurs mois plus tard, la première version était fini et fut présenté à des responsable de l'association LiberTIC.

Cependant, durant la réunion, il est apparu de nombreuses différences dans la direction du projet mené par Makina Corpus avec la direction voulu par l'association. A la fin de la réunion, il a été décidé de ne pas continuer la relation entre l'association et l'entreprise.

L'application développé par Makina Corpus est programmé en Python. Il s'agit d'une API REST qui permet aux clients d'intéragir avec les événements. Le code est basé sur le \textbf{Pyramid web framework} et \textbf{Cornice}.

Le problème de cette application est qu'elle ne répond pas aux attentes de LiberTIC par rapport à l'accessibilité et la facilité d'utilisation. En effet, il n'y a pas d'interface utilisateur disponible et l'utilisation d'une API REST est presque impossible sans connaissance en informatique.

\nb{Plus de critiques ?}

\subsubsection*{La reprise par Les Polypodes}

Suite à une réunion de l'association LiberTIC abordant l'échec de la première version, Ronan Guilloux de l'entreprise Les Polypodes, présent à cette réunion, fit la proposition suivante: Les Polypodes serait prêt à accueillir des stagiaires pour effectuer le projet. Les stagiaires seraient accueilli par l'entreprise et l'association s'engagerait à faire un suivi plus approfondit du projet pour éviter la même fin que la version précédente.

\subsubsection*{Version 2 - Les premiers stagiaires}

En janvier 2015, deux stagiaires de BTS chez Les Polypodes ont commencé à travailler sur ODE version 2.

Leur travail a été de produire un prototype d'une application sous Symfony2 permettant une gestion des événements avec un serveur CalDAV. Ils ont fait un comparatif des serveurs CalDAV et fait une analyse des sémantiques disponibles pour les événements.

Sachant que la durée de leur stage n'était que de six semaines, ils n'ont pas pu faire évoluer le prototype ( disponible à cette adresse: \url{https://github.com/polypodes/CalDAVClientPrototype} )

\nb{Expliquer un peu plus leur implication / pour différencier de la mienne}

\subsubsection*{Version 2 - Mon stage}

Depuis le 13 avril 2015, je travaille donc sur la version 2 du projet. Le but est de pouvoir réaliser le projet dans le temps qui m'est imparti ( 10 semaines ).

\subsection{Ma place sur le projet}