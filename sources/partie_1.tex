\section{Le projet}

Une des particularités de mon stage est le fait que je travaillais en collaboration direct non seulement avec l'entreprise accompagnait mon stage, mais aussi avec des réprésentants de l'association LiberTIC, clients du projet. Cette section présentera donc Les Polypodes et LiberTIC, puis, je parlerais du projet, de ses premières versions et enfin de ma place sur ce projet.

\subsection{Données Ouvertes et LiberTIC}

\begin{aquote}{Wikipédia}
``L'ouverture des données (en anglais open data) représente à la fois un mouvement, une philosophie d'accès à l'information et une pratique de publication de données librement accessibles et exploitable.''
\end{aquote}

\nb{Ici, explication personnelle de l'Open Data puis intro de LiberTIC}

\subsection{Logiciel Libre et Les Polypodes}

\begin{aquote}{Wikipédia}
``Un logiciel libre est un logiciel dont l'utilisation, l'étude, la modification et la duplication en vue de sa diffusion sont permises, techniquement et légalement.''
\end{aquote}

\nb{Ici, explication personnelle du logiciel libre puis intro des Polypodes}

J'ai effectué mon stage dans les bureaux de l'entreprise \textbf{Les Polypodes}, situés au 7\up{ème} étage de l'immeuble \textit{sigma 2000}.
Les Polypodes est une agence web créé en 2005 par \textit{M Antonio-Manuel FIDALGO} et qui emploie actuellement 8 salariés.

\nb{I need more informations}

\subsection{L'aggrégateur d'événement (ODE)}

\subsection{Les premières versions}

\subsection{Ma place sur le projet}