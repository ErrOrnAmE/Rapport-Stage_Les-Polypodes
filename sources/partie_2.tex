\section{Analyse et Conception}

Comme tous projets de développement, il faut commencer par une phase d'analyse et de conception. Le projet Open Data Event a comme particularité d'avoir déjà eu plusieurs versions avant que je ne commence à travailler dessus. C'est pourquoi la partie d'analyse des versions précédentes étais très importante.

\subsection{Mise en place de mon environnement de travail}

Il a été mis à ma disposition un \textbf{iMac} 21" avec un second écran de la même taille. J'ai donc pu découvrir l'environnement de développement offert par \textbf{OS X}. Globalement, la connaissance de l'environnement \textbf{Linux} m'aura aidé tout au long de l'utilisation d'OS X.

Après m'avoir créé un compte administrateur sur l'ordinateur fourni, j'ai pu installer tout les logiciels nécessaire à la réalisation du projet. Pour des raisons personnels, j'ai choisi d'utiliser les outils suivants:

\begin{itemize}
    \item \textbf{iTerm} (Version 2.0.0)
    \item \textbf{Sublime Text 2} (Version 2.0.2)
    \item \textbf{Google Chrome} (Version 43)
    \item \textbf{Wireshark} (Version 1.12.5)
    \item \textbf{Apple Calendar} (Version 7.0)
\end{itemize}

Pour l'utilisation de serveurs SabreDAV, Baïkal, ElasticSearch et d'autres, j'ai installé une machine virtuelle avec \textbf{Vagrant}.

\textbf{LiberTIC} étant une association militant pour les licences libres, il était évident que la majorité de mon travail (qu'il s'agisse de développement ou de compte-rendus) soit mis lui aussi sous licence libre. Ainsi, il est possible de retrouver mon travail sur \textbf{GitHub} à l'adresse suivante: \url{https://github.com/LiberTIC/ODEV2}.

En plus de mon environnement local, j'ai pu accéder à un serveur de pré-production hébergé par OVH.

\subsection{Mise à niveau préliminaire}

Dès le début de mon stage, il a été défini que le projet allait se construire sur des technologies que je ne connaissais pas et/ou ne maitrisais pas. C'est pourquoi il a été convenu que la première semaine de mon stage devais me servir pour effectuer une mise à niveau pour différentes technologies.

Premièrement, j'ai suivi un récapitulatif des commandes \textbf{Git} \rf{gitimmersion} puis lu un article sur les bonnes pratiques de l'utilisation de Git et de ses branches \rf{successfulgit}.

Ensuite, j'ai lu plusieurs articles sur les bonnes pratiques du développement \textbf{PHP} \rf{bestpracticephp} \rf{stupidvssolid}.

Enfin, j'ai lu et appliqué la totalité du \textbf{Symfony Book} \rf{symfonybook}.

\subsection{Analyse des besoins}

\subsection{Conception et première réunion}