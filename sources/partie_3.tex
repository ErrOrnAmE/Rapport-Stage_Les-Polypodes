\section{Développement et Tests}

Après la phase de Conception, viens la phase de développement, suivi de la phase de test.

\subsection{Symfony 2}

Comme expliqué précédemment, pour ce projet, j'ai utilisé \textbf{Symfony 2}. Il s'agit d'un framework PHP suivant le pattern MVC proposant de nombreux composant pour accélérer et faciliter le développement d'une application PHP. De plus, Symfony possède une communauté très impliqué, ce qui permet de trouver de très nombreuses documentation très bien rédigé.

Symfony est sponsorisé par SensioLabs, une entreprise française. Sponsorisé car, même si cette entreprise a développé la première version du framework, elle est maintenant entretenu par la communauté, en tant que projet Open-Source (\textit{MIT License}). Il est actuellement dans sa version 2.7.

Symfony offre la possibilité d'ajouter de nombreux `` Bundles '' indépendants permettant d'ajouter des fonctionnalités adaptés à nos besoin. De plus, Symfony utilise \textbf{Composer} qui est un `` Dependency Manager '' (ou gestionnaire de dépendance en français) qui permet d'ajouter un Bundle au projet en une seule commande: \textit{composer require "genemu/form-bundle"}

Symfony propose de construire une application selon la structure suivante:

\begin{figure}[h]
\begin{lstlisting}
    - app/
        - cache/
        - config/           /* Tout les fichiers de configuration
                               avec les paramètres et les routes  */
        - logs/
        - Resources/        // Les fichiers de ressources (Templates html)
        - AppKernel.php     // Base de l'application, enregistre les bundles
        - ...

    - bin/                  // Géré par Symfony, pas toucher

    - doc/                  // La documentation de l'application

    - src/                  // La plupart de notre code est ici
        - AppBundle         // Bundle principal de l'application

            - [...]         // Plus de détails à suivre

    - vendor/               /* Ici se trouve les bundles externes
                               dont Symfony                       */

    - web/                  /* Ici, nous trouvons tout les fichiers css,
                               js, les images, etc... Il s'agit du dossier
                               qui sera accessible aux utilisateurs      */

    - composer.json         /* Ce fichier liste tous les bundles que 
                               nous utilisons                       */
\end{lstlisting}
\caption{Architecture proposé par le framework Symfony}
\end{figure}

\newpage

\subsubsection*{Les composants Symfony}

Un des composants que j'ai le plus utilisé dans Symfony est: `` HTTPFoundation ''. En effet, il gère de lui même les requêtes récupérer pour les offrir dans une classe \textbf{Request} et il est possible de créer une réponse très simplement en utilisant la classe \textbf{Response}.

\begin{figure}[h]
\begin{lstlisting}[frame=single]
<?php

[...]

use Symfony\Component\HttpFoundation\Response;
use Symfony\Component\HttpFoundation\Request;

[...]

function indexAction(Request $request) {

    $name = $request->query->get('name');

    return new Response("Hello".$name);
}
[...]

\end{lstlisting}
\caption{Code simplifié avec exemple d'HTTPFoundation}
\end{figure}

Comme on peux le voir sur les quelques lignes de code précédentes, il est facile de saluer un utilisateur nous donnant son nom. Si nous voulions faire une page privé, nous pourions renvoyer un code \textbf{403 Forbidden} simplement en ajoutant 403 en argument du constructeur de Response. Simple!

Durant le développement d'ODE, j'ai utilisé de nombreux autres composants de Symfony: \textit{Controller}, \textit{FormBuilder}, \textit{Routing}, etc...

\vspace{1cm}

Pour revenir sur l'architecture proposé par Symfony (Figure 1), nous pouvions trouver le dossier AppBundle qui est prévu pour y déposer la majorité de notre code.

\nb{Change number figure}

Il est conseillé de mettre son code selon l'architecture suivante:

\subsubsection*{Backend}

Dans le dossier \textbf{Backend/}, j'ai déposé les quelques classes permettant de stocker et de retrouver des données dans la base données: \textit{CalDAV/Auth.php}, \textit{CalDAV/Principals.php}, \textit{CalDAV/Calendar.php} et \textit{Users/UserManager.php}.

Les trois premières classes étant lié à SabreDAV, je vous en parlerais dans la section suivante. La dernière, \textbf{UserManager}, est, quant à elle, lié à \textbf{FOSUSerBundle}; il s'agit d'un bundle permettant une gestion simplifié des utilisateurs. Il gère, entre autres, l'inscription, la connexion, le chiffrage des mots de passes, les sessions liés à l'utilisateurs, les rôles, etc...

\subsubsection*{Controller}

Dans ce dossier, j'ai déposé les classes de mes Controlleurs (\textit{sachant que Symfony respecte le design pattern MVC}).

La première classe, \textbf{DefaultController} est le controlleur le plus léger (2 fonctions): Index du site et page de test.

La seconde classe, \textbf{CalDAVController} est le controlleur permettant de faire le lien avec SabreDAV (plus d'explication dans la section 3.2).

La troisième, \textbf{BrowserController} est le controlleur définissant le Front-End de l'application. C'est lui qui gère les pages et les formulaires proposé à l'utilisateur. Il est notemment en lien avec FOSUserBundle pour intégrer la connexion et l'inscription des utilisateurs.

Enfin, \textbf{APIController} est le controlleur délivrant l'API. Il s'agit là de la partie de l'application délivrant la promesse de l'\textbf{Open Data}.

\subsubsection*{Entity}

Le dossier \textbf{entity/} est celui qui contient le \textbf{modèle} de l'application.

\nb{Plus de truc}

\subsubsection*{Form}

Ce dossier permet de gérer la création des formulaires en lien avec le modèle. Pour proposer des formulaires ergonomiques, j'ai utilisé un bundle, \textbf{MopaBootstrapBundle}, qui permettait d'inclure \textbf{Bootstrap} dans l'application, notemment dans la création des champs des formulaires.

\nb{Image du formulaire}

\subsubsection*{Resources}

Ce dossier est en lien avec le dossier \textit{app/Resource/}. En effet, il possède aussi des fichiers de configuration, de routes, mais aussi les librairies js et css à inclure dans le dossier \textit{web/}. Pour les besoins de l'application, j'ai écrit un fichier Javascript simple pour afficher un calendrier avec une coloration sur les dates de l'événements.

\nb{Image du calendrier}

\subsubsection*{Service}

Un service (dans Symfony) est un simple objet PHP qui remplit une fonction. Cette fonction peut être très simple: envoyer des e-mails, vérifier qu'un texte n'est pas un spam, etc. Un service est donc un objet PHP qui a pour vocation d'être accessible depuis n'importe où dans le code. Pour chaque fonctionnalité dont je peux avoir besoin dans mon application, je peux créer un ou plusieurs services.

Symfony utilise un `` ServiceContainer '' qui enregistre tout les services. Il est ainsi possible d'accèder aux services depuis les controlleurs facilement:

\begin{figure}[h]
\begin{lstlisting}[frame=single]
<?php

[...]

function indexAction(Request $request) {

    $slug = $this->get('slugify')->slugify("ExEmplE èéàó à");

    return new Response($slug);

    /* Le code précédent renverra la chaine de caractère suivante:
       exemple-eeao-a
    */
}
[...]

\end{lstlisting}
\caption{Code simplifié avec exemple d'un Service}
\end{figure}

\newpage

Le code précédent est une démonstration de l'appel au service \textbf{Slugify}, permettant de générer des slug `` URL Friendly ''.

\subsection{Intégration de SabreDAV}

Après avoir décidé d'utiliser le protocol CalDAV pour le projet, j'ai cherché une librairie PHP permettant d'implémenter CalDAV dans Symfony. Malheureusement, il n'existe pas de Bundle tout prêt pour cette utilisation. Cependant, durant ma recherche, j'ai découvert \textbf{SabreDAV}. Il s'agit d'un serveur php capable de gérer les rêquetes ``PROPPATCH'', ``MKCALENDAR'', etc. propres à CalDAV.

SabreDAV un projet Open Source développé par Fruux. Fruux fait parti du \textbf{Calendaring and Scheduling Consortium}, discutant sur le standard CalDAV, au côté d'Apple, Google, IBM et d'autres.

SabreDAV possède sa propre architecture, il a donc fallu l'intégrer dans l'application. Pour cela, j'ai créé un Controlleur spécialement pour la partie CalDAV (CalDAVController.php). Au niveau des routes, j'ai les ai toutes redirigés vers la méthode `` indexAction '' qui controlleur. Ainsi, j'ai pu transmettre la requête directement au système de SabreDAV.

Le problème que j'ai rencontré par la suite était le fait que SabreDAV gérait lui-même la constuction et l'envoi de la réponse, ce qui perturbais le fonction de Symfony.

Grâce à de nombreuses recherches sur le sujet, j'ai découvert que symfony proposait une classe, \textbf{StreamedResponse}, permettant de gérer ce cas de figure. En effet, StreamedResponse prend en argument une fonction de callback dans laquelle la réponse envoyé par SabreDAV sera attrapé par Symfony puis renvoyé vers l'utilisateur. Cela permet aussi de logger les messages renvoyé, ce qui m'a été très utile tout au long du développement de l'application.

\newpage

\begin{figure}[h]
\begin{lstlisting}[frame=single]
<?php

[...]

    public function indexAction(Request $request)
    {
        [...]

        $server = new CalDAVServer();

        $callback = function () use ($server, $request) {

            [...]

            $server->exec();

            /* These two lines log the request and the response */
            $responseBody = $server->httpResponse->getBodyAsString();
            $this->logIt($request, $server->httpResponse, $responseBody);
        };

        return new StreamedResponse($callback);
    }
[...]

\end{lstlisting}
\caption{Code simplifié avec exemple de StreamedResponse}
\end{figure}

Niveau Backend, SabreDAV a été conçu pour laisser le choix aux utilisateurs. En effet, la librairie utilise des interfaces pour les classes de Backend, ainsi, j'ai pu implémenter moi même les classes, d'abord avec ElasticSearch, puis avec PostgreSQL.

Les trois classes de Backend que j'ai implémenté sont les suivantes:

\textbf{Auth.php}: Celle ci sert à récupérer le mot de passe chiffré en fonction du nom d'utilisateur.

\textbf{Principals.php}: Cette classe permet de gérer l'accès, la création et la suppression des `` Principals ''. Il s'agit d'un concept propre à CalDAV, que l'on pourrais résumer par un dossier pour gérer les différents calendriers.

\textbf{Calendar.php}: C'est la classe la plus importante du Backend. En effet, elle rassemble toutes les fonctions pour créer, récupérer, modifier et supprimer un calendrier ou un événement. Elle s'occupe aussi des souscriptions et de lister les calendriers présents sur le serveur. J'utilise cette classe de nombreuses fois dans mes controlleurs, car même si elle est utilisé par SabreDAV, elle s'adapte très facilement à n'importe quelle utilisation.

\subsection{Implémentation avec ElasticSearch}

\textbf{ElasticSearch} est un moteur de d'indexation et de recherche. Les principaux atouts d'ElasticSearch sont sa capacité à stocker les données sous le format JSON, ainsi que sa capacité à offrir de nombreux graphiques et statistiques sur l'utilisation du service.

ElasticSearch se base sur Apache Lucene, il est donc propulsé par Java. L'installation d'ElasticSeach requiert de plus la version 1.8 de la Java Virtual Machine. ElasticSearch n'étant pas adapté à OSX, j'ai du installer une machine virtuelle sur mon Mac. Pour cela, j'ai utilisé \textbf{Vagrant}.

Après avoir installé Vagrant, il suffit d'éxécuter les commandes suivantes:

\begin{lstlisting}
vagrant init hashicorp/precise32
vagrant up
vagrant ssh
\end{lstlisting}

Et on se retrouve dans une machine virtuelle avec Ubuntu 14.04.

En ce qui concerne l'installation d'ElasticSearch sur la machine virtuelle, j'ai produit la \blurl{https://github.com/LiberTIC/ODEV2/blob/77e78f9cc36d0bc5c22eadbf2e39ab2959f10c45/doc/Elasticsearch_install.md}{Documentation de l'installation} sur le Github pour que d'autres puissent le refaire facilement.

Pour faire le lien entre Symfony et le serveur ElasticSearch, j'ai créé la classe ESManager permettant de faire des requêtes de façon simple depuis les controlleurs. De plus, je l'ai ajouté au ServiceContainer de Symfony pour l'utiliser facilement.

Un des problèmes rencontré avec Elasticsearch était le fait que l'utilisation que j'en ai fait: le stockage complet des données, posait des problèmes notamment pour l'indexation du jCal d'un événement.

\subsection{Implémentation avec PostgreSQL}

\subsection{Fondation d'une API REST}

\nb{Ne pas oublier de citer: restsymfony}

\subsection{Tests unitaires}

\subsection{Tests de comportement}